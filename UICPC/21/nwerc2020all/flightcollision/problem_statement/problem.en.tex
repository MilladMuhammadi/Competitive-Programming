\problemname{Flight Collision}

% optionally define variables/limits for this problem
\providecommand{\maxn}{}
\renewcommand{\maxn}{10^5}
\providecommand{\maxx}{}
\renewcommand{\maxx}{10^9}

\illustration{0.33}{drone.jpg}{
  \href{https://pixabay.com/images/id-1134764/}{Drone} by
  Hyeri Kim, Pixabay
}%
The \emph{Icelandic Corporation for Parcel Circulation} is the leading carrier
for transporting goods between Iceland and the rest of the world. Their newest
innovation is a drone link connecting to mainland Europe that has a number of
drones travelling back and forth along a single route.

The drones are equipped with a sophisticated system that allows them to fly
evasive manoeuvres whenever two drones come close to each other. Unfortunately,
a software glitch has caused this system to break down and now all drones are
flying along the route with no way of avoiding collisions between them.

For the purposes of this problem, the drones are considered as points moving
along an infinite straight line with constant velocity. Whenever two drones are
at the same location, they will collide, causing them to fall off their flight
path and plummet into the Atlantic Ocean. The flight schedule of the drones is
guaranteed to be such that at no point will there be three or more drones colliding at
the same location.

You know the current position of each drone as well as their velocities. Your
task is to assess the damage caused by the system failure by finding out which
drones will continue flying indefinitely without crashing.

\begin{Input}
The input consists of:
\begin{itemize}
  \item One line with an integer $n$ ($1 \leq n \leq \maxn$), the number of
    drones. The drones are numbered from $1$ to $n$.
  \item $n$ lines, the $i$th of which contains two integers $x_i$ and $v_i$
    ($-\maxx \leq x_i,v_i \leq \maxx$), the current location and the velocity
    of the $i$th drone along the infinite straight line.
\end{itemize}

The drones are given by increasing $x$ coordinate and no two drones are currently
in the same position, i.e.\ $x_i < x_{i+1}$ for each $i$. You may
assume that there will never be a collision involving three or more drones.
\end{Input}

\begin{Output}
Output the number of drones that never crash, followed by the indices of these
drones in numerically increasing order.
\end{Output}
