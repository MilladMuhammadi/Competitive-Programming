\problemname{Contest Struggles}

\providecommand{\maxn}{}
\renewcommand{\maxn}{10^6}

\illustration{0.3}{balloons.jpg}{\href{https://pixabay.com/images/id-1869790/}{Balloons} by Pexels, Pixabay}%
Lotte is competing in a programming contest. Her team has already solved $k$
out of the $n$ problems in the problem set, but as the problems become harder,
she begins to lose focus and her mind starts to wander.

She recalls hearing the judges talk about the difficulty of the problems, which
they rate on an integer scale from $0$ to $100$, inclusive. In fact, one of the
judges said that ``\textit{the problem set has never been so tough, the average
difficulty of the problems in the problem set is $d$!}''

She starts thinking about the problems her team has solved so far, and comes up
with an estimate $s$ for their average difficulty. In hope of gaining some
motivation, Lotte wonders if she can use this information to determine the
average difficulty of the remaining problems.

\begin{Input}
	The input consists of:
	\begin{itemize}
        \item One line with two integers $n$ and $k$ ($2\leq n\leq \maxn$, $0 < k < n$), the total number of
			problems and the number of problems Lotte's team has solved so far.
        \item One line with two integers $d$ and $s$ ($0\leq d,s \leq 100$), the
            average difficulty of all the problems and Lotte's estimate of the
            average difficulty of the problems her team has solved.
	\end{itemize}
\end{Input}

\begin{Output}
    Assuming Lotte's estimate is correct, output the average difficulty of the
    unsolved problems, or ``\texttt{impossible}'' if the average difficulty
    does not exist. Your answer should have an absolute or relative error of at
    most $10^{-6}$.
\end{Output}
